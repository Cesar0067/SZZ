\section{ZÁKLADNÍ VZTAHY PRO VÝPOČET CHYB V ANALOGOVÝCH OBVODECH }
Princip superpozice, celková chyba součtu a součinu dvou chybových veličin, přepočet chyb v obvodu diferenčního zapojení (výpočet vstupní napěťové nesymetrie komparátoru s BJT při známé chybě saturačního proudu vstupních tranzistorů)

\subsection{Princip superpozice}
Máme systém, který je charakterizován nějakou veličinou Q (např. offset, výstupní napětí,..). Chyba veličiny Q je dána několika dílčími nekorelovanými chybami uvnitř tohoto systému. Celková chyba veličiny Q se počítá tak, že se postupně vyjádří vliv každé dílčí chyby na veličinu Q, při tom se ostatní dílčí chyby zanedbají - položí rovno 0. Nakonec se vlivy všech dílcích chyb nekorelovaně sečtou a tím se získá celková chyba (rozptyl) veličiny Q.

\subsection{Příklad součtu výstupních proudů z proudových zrcadel}

\begin{figure}[h]
   \begin{center}
     \includegraphics[scale=0.5]{images/Chyba_Souctu.png}
   \end{center}
   \caption{Chyba součtu dvou veličin}
\end{figure}

Mějme dvě veličiny I\textsubscript{1} a I\textsubscript{2}, které jsou vzájemně nekorelované (nijak na sobě nezávisí). Proud I\textsubscript{1} nikterak nezávisí na proudu I\textsubscript{2} a naopak. Velikost těchto proudů je zatížena chybou ($\sigma$\textsubscript{1} a $\sigma$\textsubscript{2}).

Chyba $\sigma$ součtu proudů se potom vypočítá jako:
\begin{equation}
\sigma = \sqrt{\sigma_{1}^{2}+\sigma_{2}^{2}}
\end{equation}

Při sčítání nekorelovaných veličin je jedna důležitá vlastnost. Pokud je jedna veličina menší než 1/2 největší veličiny, ve výsledku se téměř neprojeví (dá se zanedbat), protože zvýší výslednou hodnotu jen asi o deetinu.

Mějme: x\textsubscript{1} = 1 a x\textsubscript{2} = 0,5. Potom:
\begin{equation}
\sigma = \sqrt{\sigma_{1}^{2}+\sigma_{2}^{2}}=\sqrt{1^{2}+0,5^{2}}=1,12\doteq 1
\end{equation}

\subsection{Příklad součinu výstupních proudů z proudových zrcadel}

\begin{figure}[h]
   \begin{center}
     \includegraphics[scale=0.5]{images/Chyba_Soucinu.png}
   \end{center}
   \caption{Chyba součinu dvou veličin}
\end{figure}

Chyba $\sigma$\textsubscript{vo} výstupního napětí Vo se spočítá jako nekorelovaný součet vlivu těchto dvou dílčích chyb. Nejdříve předpokládejme, že chyba $\sigma$\textsubscript{vr} referenčního napětí je nulová, tedy že výsledná chyba je dána pouze chybou $\sigma$A zisku A. Výstupní napětí V\textsubscript{o1} a jeho chyba $\sigma$\textsubscript{vo1} je potom:
\begin{equation}
V_{o1} = V_{r}*(A+\sigma_{A})=V_{r}*A+V_{r}*\sigma_{A}
\end{equation}

Výsledná chyba $\sigma$\textsubscript{o} výstupního napětí V\textsubscript{o} je pak dána nekorelovaným součtem výše vypočítaných dílčích chyb:
\begin{equation}
\sigma_{o1} = \sqrt{\sigma_{vo1}^{2}+\sigma_{vo2}^{2}}=\sqrt{(V_{r}+\sigma_{A})^2+(A+\sigma_{vr})^2}
\end{equation}

Vztah pro $\sigma$\textsubscript{vo} lze zapsat i následovně:
\begin{equation}
\frac{\sigma_{vo}}{A*V_{r}}=\frac{\sigma_{vo}}{V_{o}}=\sqrt{(\frac{\sigma_{vr}}{V_{r}})^2+(\frac{\sigma_{vr}}{A})^2}
\end{equation}

Tedy, že relativní normovaná chyba součinu dvou veličin zatížených chybami je dána nekorelovaným součtem relativních chyb jednotlivých složek součinu.

\subsection{Přepočet chyb v obvodu diferenčního zapojení}

\begin{figure}[h]
   \begin{center}
     \includegraphics[scale=1]{images/Prepocet.png}
   \end{center}
   \caption{Chyba vstupního proudu}
\end{figure}

Měření probíhá na teoreticky identických tranzistorech Q1 a Q2. Měřením je zjištěn rozdíl proudů (odchylky), kdy z této odchylky můžeme spočítat chybu $\sigma$I\textsubscript{1}/I\textsubscript{2} poměru proudů I\textsubscript{1} a I\textsubscript{2}:
\begin{equation}
I_{2} = I_{1}+\sigma_{1} => \frac{I_{2}}{I_{1}} = \frac{I_{1}+\sigma_{i}}{I_{1}}=1+\frac{\sigma_{i}}{I_{1}} => \frac{\sigma_{i}}{I_{1}} = \sigma_{I1/I2}
\end{equation}

Z tohoto výpočtu potom můžeme na základě úvahy "o kolik musíme změnit U\textsubscript{be} tranzistoru Q1, aby proud I\textsubscript{1} byl stejný jako proud I\textsubscript{2}" určit nesouběh Ube dvou identických tranzistorů. Jinak řečeno, určíme rozdíl U\textsubscript{be} těchto dvou tranzistorů pro případ, kdy hodnota proudu I\textsubscript{2} je přesně rovna prudu I\textsubscript{1}:
\begin{equation}
\sigma_{dUbe} = U_{T}*ln(1+\sigma_{I1/I2})
\end{equation}

\begin{figure}[h]
   \begin{center}
     \includegraphics[scale=0.4]{images/Soubeh.png}
   \end{center}
   \caption{Reálná simulace nesymetrie}
\end{figure}

























