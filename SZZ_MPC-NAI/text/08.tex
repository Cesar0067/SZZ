\section{PŘESNÝ DVOUSTUPŇOVÝ OPERAČNÍ ZESILOVAČ}
Základní koncept přesného návrhu zesilovače, vstupní bipolární stupeň, princip eliminace chyby, postup návrhu

\subsection{Základní koncept přesného návrhu zesilovače a eliminace chyby}
Popisuje strategii postupu při návrhu přesného OZ. Tato strategie se dá použít v řadě návrhů u nichž je přesnost prioritou. Přesnost OZ je většinou dána jediným parametrem - velikostí jeho vstupní napěťové nesymetrie $\sigma$\textsubscript{vos}. Vychází z toho, jakým způsobem se chová operační zesilovač se zpětnou vazbou.

\begin{figure}[h]
   \begin{center}
     \includegraphics[scale=0.5]{images/dvojOZ.png}
   \end{center}
   \caption{Blokové schéma dvojstupňového zesilovače}
\end{figure}

Předpokládejme, že první stupeň má konečnou hodnotu zisku A1 a je naprosto přesný (jeho offset je nulový). Druhý stupeň má "téměř nekonečný" zisk A2 a hodnotu offsetu $\sigma$\textsubscript{vos2}. Představme si nyní rozpojenou zpětnou vazbu a uzemněné vstupy. Na výstupu prvního stupně je nulové napětí, jako důsledek $\sigma$\textsubscript{vos1}. Na vstupu druhého stupně je potom napětí jeho nesymetrie $\sigma$\textsubscript{vos2}. Vzhledem k téměř nekonečnému zisku A2 je potom na výstupu OUT téměř nekonečně kladné napětí (v praxi omezeno U\textsubscript{cc}). 

Pokud nyní uzavřeme zpětnou vazbu, dostane se výstupní (kladné) napětí na invertující vstup. To má za následek pokles výstupního napětí prvního stupně do záporných hodnot a tato záporná hodnota výstupního napětí prvního stupně se odečítá s napětím offsetu $\sigma$\textsubscript{vos2} druhého stupně. Nakonec se celý systém ustálí ve stavu při němž první stupeň generuje napětí, jehož hodnota je stejná jako hodnota $\sigma$\textsubscript{vos2}, ale má opačnou polaritu. To je možné pouze tak, že jeho vstupní napětí má hodnotu:
\begin{equation}
V_{os} = -\frac{\sigma_{vos2}}{A1}
\end{equation}

\begin{figure}[h]
   \begin{center}
     \includegraphics[scale=0.5]{images/dvojOZ2.png}
   \end{center}
   \caption{Blokové schéma dvojstupňového zesilovače-připojená zpětná vazba}
\end{figure}

Stačí tedy navrhnout dostatečně přesný první zesilovací stupeň a tento zesilovací stupeň eliminuje vstupní chybu (offset) následujícího stupně tak, že jej převádí na svůj vstup podělená svým vlastním ziskem A1. Podle velikosti chyby $\sigma$\textsubscript{vos2} následujícího stupně se potom nastaví dostatečná (co nejmenší kvůli stabilitě) hodnota zisku A1. Běžným "nejpřesnějším" zesilovacím blokem je bipolární diferenciální zesilovač s odporovou zátěží a to přímo určuje architekturu běžných OZ.

\subsection{Vstupní bipolární stupeň}

\begin{figure}[h]
   \begin{center}
     \includegraphics[scale=0.5]{images/prvnistupen.png}
   \end{center}
   \caption{Vstupní bipolární stupeň}
\end{figure}

Transkonduktance gm1:
\begin{equation}
gm_{1} = \frac{I_{ss}}{2*U_{T}}
\end{equation}

Zisk A1:

\begin{equation}
A1 = gm_{1}*R=\frac{I_{ss}*R}{2*U_{T}}
\end{equation}

\newpage
\subsection{Postup návrhu}

První stupeň se navrhne jako přesný diferenciální zesilovač s co nejmenším vstupním offsetem $\sigma$\textsubscript{vos1} a s dostatečně velkým ziskem A1. Tento zisk A1 potom eliminuje vliv offsetu $\sigma$\textsubscript{vos2} druhého stupně. Opět se používá podmínka:
\begin{equation}
\frac{\sigma_{vos2}}{A1} = \frac{\sigma_{vos1}}{2}
\end{equation}

Vypočítáme hodnotu offsetu $\sigma$\textsubscript{vos}:
\begin{equation}
\sigma_{vos} = \sqrt{\sigma_{vos1}^2+(\frac{\sigma_{vos2}^2}{A1})^2}\leqslant1,12*\sigma_{vos1}=> \sigma_{vos} \approx \sigma_{vos1}
\end{equation}

Potom vliv offsetu druhého stupně je zanedbatelný. Jako přesný diferenciální stupeň se často používá odporově zatížený NPN stupeň, v běžných procesech má takovýto stupeň nejvyšší dosažitelnou přesnost. Jeho zisk A1 se pro běžné aplikace volí jen takový (nízký), aby stačil dostatečně na potlačení chyby $\sigma$\textsubscript{vps2} (vstupního offestu) druhého stupně. Vyšší než nezbytný zisk A1 může přinést problémy se stabilitou. Nevýhodou NPN dif. stupně může být poměrně velký vstupní odpor.

\begin{figure}[h]
   \begin{center}
     \includegraphics[scale=0.5]{images/OZ.png}
   \end{center}
   \caption{Návrh zesilovače}
\end{figure}

První stupeň má maximální možnou přesnost, která je dána offsetem $\sigma$\textsubscript{vos1}. Současně má první stupeň dostatečně vysoký zisk A1, který eliminuje vstupní chybu $\sigma$\textsubscript{vos2} druhého stupně. Celková přesnost je potom dána pouze offsetem $\sigma$\textsubscript{vos1} prvního stupně.

Dalším krokem je výpočet vstupního offsetu $\sigma$\textsubscript{vos2} druhého stupně. Ten je dán nekorelovaným součtem chyny $\sigma$\textsubscript{vos2} vlastního druhého stupně a chyby $\sigma$\textsubscript{vos1} sledovače. Chyba sledovače se skládá z chyby $\sigma$\textsubscript{voss} vlastních tranzistorů sledovače (MAL a MBL) a z chyby $\sigma$\textsubscript{vosi} nesouběhu jejich proudových zdrojů MSA a MSB.

\begin{figure}[h]
   \begin{center}
     \includegraphics[scale=0.5]{images/sled.png}
   \end{center}
   \caption{Umístění sledovače a jeho proudových zdrojů}
\end{figure}

Celková vstupní chyba $\sigma$\textsubscript{vos1} sledovače je potom spočítána jako:
\begin{equation}
\sigma_{vosl}=\sqrt{\sigma_{voss}^2+(\frac{\sigma_{voi}}{gms})^2}
\end{equation}
, kde transkonduktance MOS tranzistoru se vypočítá jako:
\begin{equation}
gm = \sqrt{2*I_{d}*kp*\frac{W}{L}}
\end{equation}

Dalším krokem je výpočet offsetu vlastního druhého stupně ($\sigma$\textsubscript{vos2}). Tato chyba je dána kombinací chyby $\sigma$\textsubscript{vos2i} diferenciálního stupně a chyby $\sigma$\textsubscript{iz} souběhu tranzistoru MZ1 a MZ2.

\begin{figure}[h]
   \begin{center}
     \includegraphics[scale=0.5]{images/offset2.png}
   \end{center}
   \caption{Výpočet offsetu vlastního druhého stupně}
\end{figure}
\pagebreak
Tato chyba se vypočítá:
\begin{equation}
\sigma_{vo2}=\sqrt{\sigma_{vo2i}^2+(\frac{\sigma_{iz}}{gms})^2}
\end{equation}

Nyní můžeme vypočítat celkový vstupní offset $\sigma$\textsubscript{vos2} celého druhého stupně (vlastní druhý stupeň a sledovač) jako nekorelovaný součet chyby $\sigma$\textsubscript{vos2} vlastního druhého stupně a chyby $\sigma$\textsubscript{vos1} sledovače:
\begin{equation}
\sigma_{vos2}=\sqrt{\sigma_{vo2}^2+\sigma_{vosl}^2}
\end{equation}

Posledním krokem je návrh zisku A1 prvního stupně. Tento stupěň má chybu $\sigma$vos1. Druhý stupeň má chybu (offset) $\sigma$\textsubscript{vos2}. Tato chyba je eliminována ziskem A1 - pro zanedbání offsetu $\sigma$\textsubscript{vos2} druhého stupně musí platit
\begin{equation}
\frac{\sigma_{vos2}}{A1}\leq\frac{\sigma_{vos1}}{2} => A1\geq2*\frac{\sigma_{vos2}}{\sigma_{vos1}}
\end{equation}

Napěťový zisk A1 odporově zatíženého bipolárního diferenciálního stupně je dán poměrem úbytku na zatěžovacích odporech a teplotního napětí U\textsubscript{T}:
\begin{equation}
A1=\frac{I_{c}*R_{c}}{U_{T}} => R_{c}=\frac{A1*U_{T}}{I_{C}}
\end{equation}

Celková hodnota vstupní chyby $\sigma$\textsubscript{vos} (offsetu) se potom spočítá jako:
\begin{equation}
\sigma_{vos}=\sqrt{\sigma_{vos1}^2+(\frac{\sigma_{vos2}}{A1})^2}
\end{equation}


















