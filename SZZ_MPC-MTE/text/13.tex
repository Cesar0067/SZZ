\section{Ochrana proti ESD}
-vysvětlete důvod ochrany polovodičových součástek, popište
specifika při práci s citlivými součástkami, balení a skladování


\subsubsection{Vznik elektrostatického výboje}

Hlavní příčina je nahromadění elektrického náboje způsobené nerovnováhou elektronů na
povrchu daného materiálu, jež vytváří měřitelné elektrické pole. Elektrostaticky nabitým
materiálem můžeme nazvat ten materiál, který má buď kladný, nebo záporný náboj.

Samotný výboj vzniká tehdy, vytvoří-li se elektrostatické napětí mezi předmětem a jeho okolí, a poté nastane spontánní výboj v podobě elektrického proudu.
\begin{equation}
Q= I*t
\end{equation}

Elektrostatický výboj je miniaturní záblesk elektrostatického náboje, který přechází z jedné
desky na druhou (překonání dialektrické pevnosti vzduchu).

Elektrostatický výboj a tím poškození polovodičových obvodů může vzniknout čtyřmi způsoby:
\begin{itemize}
\item dotknutím se polovodičového obvodu (pinů), jsme-li elektrostaticky nabiti
\item dotknutím se nabitého polovodičového obvodu například uzemněné plochy
\item dotknutím se například nabitého nářadí polovodičového obvodu
\item ocitne-li se obvod náhodou v elektrostatickém poli
\end{itemize}


\subsection{Důvod ochrany polovodičových součástek}
Poškození vlivem velkého proudu:
\begin{itemize}
\item Roztavení vodivých spojů nebo rezistorů.
\item Roztavení polovodičových přechodů.
\item Poškození kontaktů mezi polovodičem a spoji.
\end{itemize}
Poškození vlivem velkého napětí:\\
Průraz dielektrické izolace.

\textbf{Sled událostí vedoucí ke zničení PN přechodu:}
\begin{itemize}
\item Napětí na závěrně polarizovaném PN přechodu překročí napětí lavinového průrazu.
\item Druhý průraz může nastat, pokud dojde průchodem proudu k ohřevu PN přechodu
na teplotu, kdy dochází vlivem tepelné generace nosičů náboje k lavinovému
průrazu.
\item Místem druhého průrazu protéká velmi velký proud, který způsobuje velké lokální
zahřívání materiálu.
\item Tím se urychluje termální generace nosičů náboje (další lokální zvyšování proudu),
což vede ke kladné zpětné vazbě. Vyvrcholením je roztavení křemíku, při překročení
teploty 1415 $^{\circ}$C.
\item Pokud je teplota dostatečná k roztavení kovového vodiče v oblasti kontaktu, může se roztavený kov dostat do oblasti polovodičového přechodu a tím způsobit odporový zkrat polovodičového přechodu
\end{itemize}

\textbf{Sled událostí vedoucí k průrazu dielektrika:}
\begin{itemize}
\item Závěrné napětí dielektrika je překročeno (obvykle na hraně dielektrické vrstvy).
\item Tímto bodem poté proteče relativně velmi velký proud, který vyvolá místní přehřátí.
\item V tomto místě se vytvoří přetavený amorfní nebo polykrystalický křemík.

\end{itemize}

\subsection{Specifika při práci s citlivými součástkami}
Navrhovat zařízení s ohledem na jejich odolnost vůči ESD (použití méně citlivých
součástek, nebo ochrana součástky na vstupu, na deskách a v zařízeních).

Definovat si potřebnou úroveň ochrany u výrobku a prostoru, kde ho vyrábím (zjištění
odolnosti používaných součástek, stav výrobních zařízení a prostor).

Zjistit a definovat vyhrazené prostory EPA (všechny vodivé a rozptylující materiály včetně
personálu jsou propojeny na společný zemní potenciál).

Vyloučit, nebo snížit vytváření a kumulaci elektrostatického náboje.

Odstranění všech materiálů, které mohou při manipulaci generovat, nebo akumulovat
statický náboj (krabice, plastické obaly, lahve, kelímky atd.)

Udržování zařízení na stejných elektrostatických potenciálech, nejlépe potenciálu
země

Uzemnění osob přes náramky, propojit pracovní povrchy a podlahy se zemí



\subsection{Balení a skladování}
Používat ochranné obaly

Stahovací a smršťovací fólie vycházejí z materiálů používaných pro antistatické sáčky (polyolefin)

\textbf{Antistatické boxy, krabičky z plastů}\\
Jejich výhodou je jednoznačně mnohonásobné použití a výborné mechanické
vlastnosti.\\
Antistatická úprava je ve většině případů celoobjemová.\\
Výsledná rezistivita je dána složením granulátu při vstřikování nebo přísadami do
základního materiálu při výrobě.\\

\textbf{Antistaticky ošetřené lepenkové krabice}\\
Antistaticky ošetřená lepenka je klasická třívrstvá lepenka, na kterou je oboustraně
aplikována nejprve elektrostaticky vodivá (stínící) vrstva a poté druhá dissipativní vrstva.\\
Odstínění vnějších elektrostatických polí a zároveň ochrana elektrostaticky citlivých součástek a dílů.

\textbf{Antistatické pěny}\\
Používá se v kombinaci s dalším obalem např. antistaticky upravenou lepenkou.\\
Největší výhodou je absorpce mechanických rázů a vibrací\\
Měkčí polyuretanové pěny jsou vhodnější pro výplně a větší díly. Naopak tvrdší polyetylenové pěny lépe chrání malé součástky a díly – kusové balení (zapíchávání).\\

\textbf{Trubice pro integrované obvody}
Jsou vyráběny z elektrostatických plastů.\\
Bývají průhledné, aby bylo možné identifikovat a přepočítat jejich obsah.\\
Pro přepravu od výrobce k zákazníkovi se balí ještě do ochraných antistatických
sáčků či přepravek.\\

\textbf{Kotouče a pásy pro ECS}\\
Kotouče a pásy na elektrostaticky citlivé součástky pro povrchovou montáž jsou
vyráběny z vodivých plastů.\\
Jsou určeny pro přepravu pasivních součástek, diskrétních polovodičových prvků
(diody, tranzistory, LED,...) a jednoduchých IO v pouzdrech SMD.\\

\textbf{Antistatické sáčky}
V této oblasti je velmi široký, výběr se řídí potřebami
uživatele.\\
Existuje také mnoho způsobů uzavírání (sponkování, přelepení páskou a tepelné svařování).\\

















